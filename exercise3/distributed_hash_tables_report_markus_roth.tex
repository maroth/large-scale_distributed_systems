\documentclass[a4paper]{article}
\nonstopmode

\usepackage{listings}
\usepackage[T1]{fontenc}
\usepackage{color}
\usepackage{pxfonts}
\usepackage{pdfpages}

\definecolor{dkgreen}{rgb}{0,0.6,0}
\definecolor{gray}{rgb}{0.5,0.5,0.5}
\definecolor{mauve}{rgb}{0.58,0,0.82}


\lstset{frame=tb,
    aboveskip=3mm,
    belowskip=3mm,
    showstringspaces=false,
    columns=flexible,
    basicstyle={\small\ttfamily},
    numbers=none,
    numberstyle=\tiny\color{gray},
    keywordstyle=\color{blue},
    commentstyle=\color{dkgreen},
    stringstyle=\color{mauve},
    breaklines=true,
    breakatwhitespace=true,
    tabsize=3
}

\title{Distributed Hash Tables}

\author{Markus Roth}

\begin{document}

\maketitle

\tableofcontents

\section{Solution Overview}

\subsection{Constructing a Ring}

\begin{figure}
    \input{task_2_1/search_performance}
    \label{fig:2-1}
\end{figure}

Every node of the ring sends 500 queries for random keys. As we can see the figure \ref{fig:2-1}, the number of hops is evenly distributed between 0 and 63 in a ring of size 64. This is as expected: Since the nodes are randomly distributed around the ring, and the search only takes place in a linear fashion, the search will have on average n/2 hops, and every possible number of hops from the minimum to the maximum will have the same chance of occuring.

\end{document}
